\documentclass[ignorenonframetext, professionalfonts, hyperref={pdftex, unicode}]{beamer}

\usetheme{Copenhagen}
\usecolortheme{wolverine}

\usepackage[orientation=landscape, size=custom, width=16, height=9.75, scale=0.5]{beamerposter}	

%Packages to be included

\usepackage{textcomp}

\usepackage[russian]{babel}
\usepackage[utf8]{inputenc}
\usepackage[T1]{fontenc}

\usepackage{beamerthemesplit}

\usepackage{ulem}

\usepackage{verbatim}

\usepackage{ucs}
\usepackage{listings}
\lstloadlanguages{C, make, bash}

\lstset{escapechar=`,
	extendedchars=false,
	language=C, 
	tabsize=2, 
	columns=fullflexible, 
%	basicstyle=\scriptsize,
	keywordstyle=\color{blue}, 
	commentstyle=\itshape\color{brown},
%	identifierstyle=\ttfamily, 
	stringstyle=\mdseries\color{green}, 
	showstringspaces=false, 
	numbers=left, 
	numberstyle=\tiny, 
	breaklines=true, 
	inputencoding=utf8x,
	keepspaces=true,
	morekeywords={u\_short, u\_char, u\_long, in\_addr}
	}

\definecolor{darkgreen}{cmyk}{0.7, 0, 1, 0.5}

\lstdefinelanguage{diff}
{
    morekeywords={+, -},
    sensitive=false,
    morecomment=[l]{//},
    morecomment=[s]{/*}{*/},
    morecomment=[l][\color{darkgreen}]{+},
    morecomment=[l][\color{red}]{-},
    morestring=[b]",
}



%%%%%%%%%%%%%%%%%%%%%%%%%%%%%%%%%%%%%%%%%%%%%%%%%
%%%%%%%%%% PDF meta data inserted here %%%%%%%%%%
%%%%%%%%%%%%%%%%%%%%%%%%%%%%%%%%%%%%%%%%%%%%%%%%%
\hypersetup{
	pdftitle={Введение в GNU/Linux},
	pdfauthor={Epam/LLPD}
}





%%%%%% Beamer Theme %%%%%%%%%%%%%

	
\title{Введение в GNU/Linux}
\author{Epam/LLPD}



%%%%%%%%%%%%%%%%%%%%%%%%%%%%%%%%%%%%%%%%%%%%%%%%%
%%%%%%%%%% Begin Document  %%%%%%%%%%%%%%%%%%%%%%
%%%%%%%%%%%%%%%%%%%%%%%%%%%%%%%%%%%%%%%%%%%%%%%%%




\begin{document}

\section{Введение}


\frame{
	\frametitle{}
	\titlepage
	\vspace{-0.5cm}
	\begin{center}
	%\frontpagelogo
	\end{center}
}
\frame{
	\tableofcontents
%	[hideallsubsections]
}






%%%%%%%%%%%%%%%%%%%%%%%%%%%%%%%%%%%%%%%%%   
%%%%%%%%%% Content starts here %%%%%%%%%%
%%%%%%%%%%%%%%%%%%%%%%%%%%%%%%%%%%%%%%%%%









\subsection{Введение}

\begin{frame}{Цель курса}
	\begin{center}
		\Huge
		Увеличение популярности GNU/Linux среди программистов.

		\hrulefill

		\normalsize
		Воспитание потенциальных сотрудников ;-)
	\end{center}
\end{frame}


\begin{frame}{Состав курса}
	\begin{itemize}
		\item Представление об архитектуре GNU/Linux дистрибутива
			\pause
		\item "Ежедневные" навыки работы в консоли
			\pause
		\item Введение в shell-программирование
			\pause
		\item Работа с классическими средствами разработки
			\pause
		\item Все, чего вы не знали и боялись спросить
	\end{itemize}
\end{frame}




\subsection{GNU/Linux}

\begin{frame}{Терминология}
	\begin{itemize}
		\item GNU -- GNU's Not Unix!
		\begin{itemize}
			\item 1983. Ричард Столлман. Свободное ПО.
		\end{itemize}

		\pause

		\item POSIX
		\begin{itemize}
			\item 1988. Portable Operating System Interface for Unix. 
		\end{itemize}

		\pause

		\item Linux
		\begin{itemize}
			\item 1991. Линус Торвальдс. Ядро.
		\end{itemize}

	\end{itemize}
\end{frame}

\begin{frame}{Лицензии: открытые и свободные}

	\begin{block}{ Р.Столлман: 4 свободы}

		\begin{itemize}
			\item Свобода 0: Свобода запускать программу в любых целях.
			\item Свобода 1: Свобода изучения работы программы и адаптация её к вашим нуждам. 
				Доступ к исходным текстам является необходимым условием.
			\item Свобода 2: Свобода распространять копии,  так что вы можете помочь вашему товарищу.
			\item Свобода 3: Свобода улучшать программу и публиковать ваши улучшения,
				так что всё общество выиграет от этого.
				Доступ к исходным текстам является необходимым условием.
		\end{itemize}
	\end{block}


\end{frame}

\begin{frame}{Copyleft }

	\begin{block}{ \textcopyleft  -- ``Копилефт''}
	Авторское лево -- концепция и практика использования законов авторского права для обеспечения 
	невозможности ограничить любому человеку право использовать,  изменять и распространять как 
	исходное произведение,  так и произведения,  производные от него.
	\end{block}


	При копилефте все производные произведения должны распространяться под той же лицензией,
	что и оригинальное произведение.

\end{frame}


\begin{frame}{Лицензии}
	\begin{itemize}
		\item GPL
		\item LGPL
		\item AGPL
		\item BSD
		\item Creative Commons *
		\item MIT
		\item Mozilla Public License
	\end{itemize}
\end{frame}




\subsection{Дистрибутивы ОС Linux}




\end{document}
