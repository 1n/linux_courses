\documentclass[ignorenonframetext, professionalfonts, hyperref={pdftex, unicode}]{beamer}

\usetheme{Copenhagen}
\usecolortheme{wolverine}

\usepackage[orientation=landscape, size=custom, width=16, height=9.75, scale=0.5]{beamerposter}	

%Packages to be included

\usepackage{textcomp}

\usepackage[russian]{babel}
\usepackage[utf8]{inputenc}
\usepackage[T1]{fontenc}

\usepackage{beamerthemesplit}

\usepackage{ulem}

\usepackage{verbatim}

\usepackage{ucs}
\usepackage{listings}
\lstloadlanguages{C, make, bash}

\lstset{escapechar=`,
	extendedchars=false,
	language=C, 
	tabsize=2, 
	columns=fullflexible, 
%	basicstyle=\scriptsize,
	keywordstyle=\color{blue}, 
	commentstyle=\itshape\color{brown},
%	identifierstyle=\ttfamily, 
	stringstyle=\mdseries\color{green}, 
	showstringspaces=false, 
	numbers=left, 
	numberstyle=\tiny, 
	breaklines=true, 
	inputencoding=utf8x,
	keepspaces=true,
	morekeywords={u\_short, u\_char, u\_long, in\_addr}
	}

\definecolor{darkgreen}{cmyk}{0.7, 0, 1, 0.5}

\lstdefinelanguage{diff}
{
    morekeywords={+, -},
    sensitive=false,
    morecomment=[l]{//},
    morecomment=[s]{/*}{*/},
    morecomment=[l][\color{darkgreen}]{+},
    morecomment=[l][\color{red}]{-},
    morestring=[b]",
}



%%%%%%%%%%%%%%%%%%%%%%%%%%%%%%%%%%%%%%%%%%%%%%%%%
%%%%%%%%%% PDF meta data inserted here %%%%%%%%%%
%%%%%%%%%%%%%%%%%%%%%%%%%%%%%%%%%%%%%%%%%%%%%%%%%
\hypersetup{
	pdftitle={Введение в GNU/Linux},
	pdfauthor={Epam/LLPD}
}





%%%%%% Beamer Theme %%%%%%%%%%%%%

	
\title{Введение в GNU/Linux}
\author{Epam/LLPD}



%%%%%%%%%%%%%%%%%%%%%%%%%%%%%%%%%%%%%%%%%%%%%%%%%
%%%%%%%%%% Begin Document  %%%%%%%%%%%%%%%%%%%%%%
%%%%%%%%%%%%%%%%%%%%%%%%%%%%%%%%%%%%%%%%%%%%%%%%%




\begin{document}

\section{Командная строка}


\frame{
	\frametitle{}
	\titlepage
	\vspace{-0.5cm}
	\begin{center}
	%\frontpagelogo
	\end{center}
}
\frame{
	\tableofcontents
%	[hideallsubsections]
}

\subsection{Командная оболочка}
\begin{frame}[fragile]{Определение(не совсем формальное)}
\textbf{Shell} -- приложение, обеспечивающее выполнение других приложений и их взаимодействие, а также представляющая услуги командной строки. 
\begin{center}
  или
\end{center}
\textbf{Shell} -- приложение, обеспечивающее доступ к основным функциям ядра.

\pause
\vspace{0.5in}
Пример shell из Windows-world -- cmd.exe
\vspace{0.5in}

Минимальный дистрибутив Linux -- ядро + shell 

\end{frame}
\begin{frame}[fragile]{Основные типы shell в Unix}
  \begin{itemize}
    \item Bourne shell совместимые
      \begin{itemize}
        \item \textbf{sh} исходная bourne shell (Steve Bourne, 1978)
        \item \textbf{ksh} Korn shell (David Korn, 1983)
        \item \textbf{ash} $[$BSD$]$ Almquist shell (Kenneth Almquist,1989)  
        \item \textbf{bash} $[$GPL$]$ Bourne-again shell (Brian Fox, 1989)
        \item \textbf{zsh} $[$BSD$]$ Z shell (Paul Falstad,1990)
        \item \textbf{/bin/sh} Указывает на POSIX-совместимую shell
      \end{itemize}
  \item C shell совместимые
      \begin{itemize}
        \item \textbf{csh}  Исходная С shell (Bill Joy, 1978)
        \item \textbf{tcsh} $[$BSD$]$ TENEX C shell (Ken Greer, 1981)
       \end{itemize}
  \end{itemize}
\end{frame}
\begin{frame}[fragile]{Маленькое упражнение}
\begin{lstlisting}[language=bash]
cat /etc/shells
ls -l <filename> # для каждого элемента /etc/shells
readlink -e <filename> 
\end{lstlisting}
\end{frame}
\subsection{bash -- Работа в командной строке}
\begin{frame}[fragile]{Получение помощи}
  \begin{itemize}
    \pause
    \item \textbf{man} - помощь по внешним командам
    \pause
    \item \textbf{help} - помощь по внутренним командам bash (также man bash)
    \pause
    \item \textbf{info} - расширенная помощь по некоторым командам (texinfo format)
      \begin{itemize}
       \item   Попробовать {\tt info coreutils}
       \item   Справка по навигации -- нажать h
      \end{itemize}
  \end{itemize}
\end{frame}
\begin{frame}[fragile]{Основное о man}
\begin{columns}
  \column{2.2in}
  \begin{itemize}
        \item Прочитайте {\tt man man} !
        \item Apropos {\tt man -k <слово>}
        \item Разделы (sections)
          \begin{itemize}
            \item[1] Основная секция(юзерские программы) 
            \item[2] Syscalls
            \item[3] С library
            \item[5] Конфигурационные файлы
            \item[8] Системные службы
          \end{itemize}
  \end{itemize}
  \textbf{Замечание}

  Обычно внутри страницы работает поиск с помощью '/'
 \pause 
 \column{1in}
 \begin{block}{Попробовать}
\begin{lstlisting}
man -k printf
man 3 printf
man 1 printf
man -a printf
\end{lstlisting}
\end{block}
\end{columns}
\end{frame}
\begin{frame}[fragile]{Базовые команды}
\begin{columns}
  \column{0.7\textwidth}
  \begin{itemize}
    \item Навигация по файловой системе
      \begin{itemize}
        \item Список файлов в (текущей по умолчанию) директории \pause \textbf{ls} (man ls)
          \begin{itemize}
            \item {\tt ls -l} Расширенная информация о файлах
          \end{itemize}
        \pause
        \item Смена текущей директории \pause \textbf{cd} (help cd)
        \item Имя текущей директории \pause \textbf{pwd} (help pwd)
      \end{itemize}
      \pause
    \item Работа с файлами и директориями
      \begin{itemize}
        \item Создание пустого файла \textbf{touch} (man touch)
        \item Копирование файла \textbf{cp} (man cp)
        \item Удаление файла \textbf{rm} (man rm) \textbf{DANGER!!!}
          \begin{itemize}
            \item Опции {\tt -r -f -i}  
          \end{itemize}
        \item Создание директории \textbf{mkdir} (man mkdir)
            \begin{itemize}
              \item Опция {\tt -p}
            \end{itemize}
        \item Удаление директории \textbf{rmdir} (man rmdir)
      \end{itemize}
  \end{itemize}
  \column{0.25\textwidth}
   \begin{block}{Упражнение}
     Создать иерархию директорий
\begin{lstlisting}
dir1/dir1.1/dir1.1.1
dir1/dir1.2/dir1.2.1
dir1/dir1.2/dir1.2.2
\end{lstlisting}

Внутри каждой создать файл

Удалить все созданное
   \end{block}
 \end{columns}
\end{frame}
\begin{frame}{Важные аббревиатуры внутри командной строки}
  \begin{itemize}
    \item Для директорий
    \begin{itemize}
      \item {\tt ~} Домашняя директория
      \item {\tt ~<username>} Домашняя директория пользователя
      \item {\tt ../} Родительская директория
    \end{itemize}
  \pause  
  \item Wildcards
    \begin{itemize}
      \item {\tt *} Любой набор символов {\tt file*txt : file1.txt filefilefiletxt}
      \item {\tt $[$<список>$]$ } символ из заданного набора {\tt }
      \item {\tt ?} любой один символ
     \end{itemize}

  \end{itemize}
\end{frame}       
\begin{frame}{Горячие клавиши}
  \begin{itemize}
    \item \textbf{Tab} -- дополнение текущей команды
     \pause
    \item История команд
      \begin{itemize}
        \item Клавиши курсора -- навигация по истории
        \item Ctrl-R -- поиск в истории по фрагменту
        \item Ctrl-O (после выполнения вставить следующую команду из истории)
        \item Команда {\tt history}
       \end{itemize}
    \item Навигация
   \end{itemize}
\end{frame}

\section{Взаимодействие процессов}
\begin{frame}{Unix way}
  \textbf{Цель} -- максимальная модульность: большое количество простых приложений, взаимодействующих друг с другом для решения задач
  \begin{itemize}
    \item Каждое приложение открывает 3 стандартных файловых дескриптора stdin (fd 0), stdout(fd 1), stderr (fd 2)
    \item Приложения могут работать как фильтр из STDIN в STDOUT, можно объединять несколько приложений в конвейер
    \item Синтаксис {\tt <app1> | <app2>}
  \end{itemize}
\end{frame}
\begin{frame}{Дополнительный набор команд}
 \begin{itemize}
  \item {\tt cat} - Вывод файла в stdout, соединение нескольких файлов в stdout
  \item {\tt wc} - подсчет статистики символов в файле или в stdin 
    \begin{itemize}
      \item[-l] считать только строки
    \end{itemize}
  \item {\tt sort} - сортировка строк файла
    \begin{itemize}
      \item[-n] -по числам в начале строки
      \item[-r] - обратный порядок
    \end{itemize}
  \item {\tt uniq} - объединение одинаковых строк в одну
    \begin{itemize}
      \item[-c] считать одинаковые строки
      \item[-d] выводить только повторяющиеся строки
    \end{itemize}
  \item {\tt tr} - замена набора символов
  \item {\tt less} - программа-пейджер
 \end{itemize}
\end{frame}
\begin{frame}{Некоторые примеры использования}
\begin{lstlisting}[language=bash]
ls  | wc -l # подсчет числа файлов
man uniq | tr  '[:space:]' '\n' | sort | uniq -c | sort -n | less # подсчет количества слов в тексте man uniq
history | wc -l # подсчет ранее введенных команд
\end{lstlisting}
\end{frame}
\begin{frame}{Кастомизация bash}
\end{frame}
\end{document}
