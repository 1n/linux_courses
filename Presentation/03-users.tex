
\documentclass[ignorenonframetext, professionalfonts, hyperref={pdftex, unicode}]{beamer}

\usetheme{Copenhagen}
\usecolortheme{wolverine}

\usepackage[orientation=landscape, size=custom, width=16, height=9.75, scale=0.5]{beamerposter}	

%Packages to be included

\usepackage{textcomp}

\usepackage[russian]{babel}
\usepackage[utf8]{inputenc}
\usepackage[T1]{fontenc}

\usepackage{beamerthemesplit}

\usepackage{ulem}

\usepackage{verbatim}

\usepackage{ucs}
\usepackage{listings}
\lstloadlanguages{C, make, bash}

\lstset{escapechar=`,
	extendedchars=false,
	language=C, 
	tabsize=2, 
	columns=fullflexible, 
%	basicstyle=\scriptsize,
	keywordstyle=\color{blue}, 
	commentstyle=\itshape\color{brown},
%	identifierstyle=\ttfamily, 
	stringstyle=\mdseries\color{green}, 
	showstringspaces=false, 
	numbers=left, 
	numberstyle=\tiny, 
	breaklines=true, 
	inputencoding=utf8x,
	keepspaces=true,
	morekeywords={u\_short, u\_char, u\_long, in\_addr}
	}

\definecolor{darkgreen}{cmyk}{0.7, 0, 1, 0.5}

\lstdefinelanguage{diff}
{
    morekeywords={+, -},
    sensitive=false,
    morecomment=[l]{//},
    morecomment=[s]{/*}{*/},
    morecomment=[l][\color{darkgreen}]{+},
    morecomment=[l][\color{red}]{-},
    morestring=[b]",
}



%%%%%%%%%%%%%%%%%%%%%%%%%%%%%%%%%%%%%%%%%%%%%%%%%
%%%%%%%%%% PDF meta data inserted here %%%%%%%%%%
%%%%%%%%%%%%%%%%%%%%%%%%%%%%%%%%%%%%%%%%%%%%%%%%%
\hypersetup{
	pdftitle={Введение в GNU/Linux},
	pdfauthor={Epam/LLPD}
}





%%%%%% Beamer Theme %%%%%%%%%%%%%

	
\title{Введение в GNU/Linux}
\author{Epam/LLPD}



%%%%%%%%%%%%%%%%%%%%%%%%%%%%%%%%%%%%%%%%%%%%%%%%%
%%%%%%%%%% Begin Document  %%%%%%%%%%%%%%%%%%%%%%
%%%%%%%%%%%%%%%%%%%%%%%%%%%%%%%%%%%%%%%%%%%%%%%%%




\begin{document}

\frame{
	\frametitle{}
	\titlepage
	\vspace{-0.5cm}
	\begin{center}
	%\frontpagelogo
	\end{center}
}
\frame{
	\tableofcontents
%	[hideallsubsections]
}
\begin{frame}{Многопользовательская модель}   
 \begin{itemize}
   \item Linux -- многопользовательская система
   \item Привилегии пользователей
     \begin{itemize}
       \item root
       \item other users
      \end{itemize}
     \end{itemize}
\end{frame}
\section{Механизмы разделения привилегий}
\subsection{Классический UNIX}
\begin{frame}{Пользователи, группы и файлы}
\begin{itemize}
  \item Каждый пользователь принадлежит одной или нескольким \textbf{группам}
  \item Каждый файл и директория принадлежит
    \begin{itemize}
      \item Одному пользователю 
      \item Одной группе
    \end{itemize}
  \pause
  \item  Разрешения что либо делать с файлом определяются по отношению к
    \begin{enumerate}
      \item Пользователю-владельцу файла
      \item Группе владеющей файлом
      \item Всем остальным пользователям
    \end{enumerate}

\end{itemize}
\pause
\begin{columns}
  \column{0.48\textwidth}
  \begin{itemize}
    \item {\tt ls -l} 3,4 поле 
    \item {\tt groups}
   \end{itemize}
  \column{0.48\textwidth}
  \begin{block}{Попробовать}
    {\tt ls -l /usr/bin/}

    {\tt groups}

    {\tt groups root}
  \end{block}
\end{columns}
\end{frame}

\begin{frame}{Типы разрешений для файлов}
\begin{columns}
\column{0.48\textwidth}
\begin{center}
  \textbf{Разрешения для файла}
\end{center}
\begin{itemize}
  \item Три типа разрешений
    \begin{enumerate}
      \item чтение read(r)
      \item запись write(w)
      \item выполнение execute(x)
    \end{enumerate}
\end{itemize}
\column{0.48\textwidth}
\begin{center}
  \textbf{Разрешения для директорий}
\end{center}
\begin{itemize}
  \item Три типа разрешений
    \begin{enumerate}
      \item поиск файлов в директории read(r) 
      \item добавление и удаление файлов write(w)
      \item заход в директорию execute(x)
    \end{enumerate}
\end{itemize}
\end{columns}

\pause

Попробовать {\tt ls -l /usr/bin}

\pause

Пересчет мнемонического разрешения в битовую маску 

$r\to4, w\to2 , x\to1$ 

rwxrw-r-x$\to$765
\end{frame}

\begin{frame}{Команды для управления пользователями и разрешениями файлов}
\begin{columns}
 \column{0.48\textwidth}
 \begin{itemize}
   \item {\tt chown}
   \item {\tt chmod}
 \end{itemize}
 \column{0.48\textwidth}
 \begin{itemize}
   \item {\tt useradd, usermod, userdel}
   \item {\tt groupadd, groupmod, groupdel}
   \item {\tt su, sudo}
 \end{itemize}
\end{columns}
\end{frame}

\begin{frame}
\begin{block}{Упражнения}
  \begin{enumerate}
   \item Создать директорию без r разрешения но с x разрешением, внутри нее создать поддиректорию с rwx разрешениями (для пользователя altlinux)
   \item Создать нового пользователя testuser.
   \item Скопировать {\tt /bin/bash} (под именем mysh) в домашнюю директорию пользователя altlinux  и поставить r-x разрешение только для other
   \item Попробовать выполнить скопированный файл от имени пользователя altlinux, затем от имени пользователя testuser
   \item Создать новую группу testgroup
   \item Изменить группу владеющую mysh на testgroup и сделать {\tt chmod 464 mysh}
   \item Попробовать выполнить mysh от имени altlinux и root. 
   \item Добавить пользователя altlinux в группу testgroup и попробовать выполнить mysh еще раз
   \item Получить список групп которым принадлежат устройства в {\tt /dev}
  \end{enumerate}
\end{block}
\end{frame}

\begin{frame}{SUID программы}
 \begin{block}{Попробовать}
   {\tt id}

   {\tt ls -l `which su`}
 \end{block}
 \pause
 \begin{itemize}
   \item Некоторые программы должны выполняться от имени обычного пользователя, но иметь больше привилегий
   \item Для этого у них устанавливается suid или sgid биты
   \item Установка suid (например {\tt chmod 4710 <file>})
 \end{itemize}
 \pause
 \begin{block}{Упражнение}
   \begin{itemize}
     \item Под root создать копию утилиты {\tt id} (назвать, например, {\tt id2}) в директории /usr/bin/
     \item Установить suid бит для этой утилиты
     \item Запустить {\tt id2} от имени пользователя altlinux
     \item То же с sgid битом
    \end{itemize}
 \end{block}
\end{frame}

\begin{frame}{Опасности SUID}
  \begin{itemize}
    \item Возможность backdoor через suid программу
      \begin{itemize}
        \item Shell игнорирует effective uid
        \item Скрипты обычно тоже игнорируют
        \item nosuid mount option
       \end{itemize}
     \item Атака через buffer overflow в существующей suid программе
       \begin{itemize}
         \item не использовать strcpy, sprintf, ... в security critical
         \item А если все же не уследили
           \begin{itemize}
             \item рандомизация стека
             \item grsecurity
             \item частично selinux
           \end{itemize}
       \end{itemize}
   \end{itemize}
\end{frame}


\begin{frame}{SUID, SGID и sticky bit для директорий}
  \begin{itemize}
    \item sgid для директорий -- все поддиректории и файлы внутри имеют тот же group id
    \item suid -- игнорируется
    \item Зато есть sticky bit 
   \end{itemize}
\end{frame}

\begin{frame}{Дополнительные атрибуты файлов(ext2,ext3 filesystem)}
  \begin{itemize}
    \item Команды
      \begin{itemize}
        \item chattr
        \item lsattr
       \end{itemize}
     \item Важные дополнительные атрибуты
       \begin{itemize}
         \item i (immutable)
         \item A (noatime)
         \item a (append-only)
         \item S (sync)
        \end{itemize}
    \end{itemize}
\end{frame}

\begin{frame}{Хранение информации о пользователях в системе}
  \begin{itemize}
    \item user id, group id -- это числа
    \item Утилита id 
    \item {\tt /etc/passwd} {\tt man 5 passwd}
    \item {\tt /etc/group}  {\tt man 5 group}
    \item {\tt /etc/shadow} -- многие системы, но не Altlinux
    \item {\tt /etc/tcb/<username>/shadow} altlinux
    \item Утилита {\tt passwd}
  \end{itemize}
  \pause
  \begin{block}{Практика}
    \begin{itemize}
       \item Добавить пользователя и группу и посмотреть за изменениями в перечисленных файлах
       \item[*] Домашнее упражнение: создать пользователя без использования системных утилит, редактируя passwd, group, shadow
    \end{itemize}
  \end{block}
\end{frame}


\begin{frame}{Процесс аутентификации пользователя}
  \begin{itemize}
    \item Программы, которые зависят от правильной аутентификации
      \begin{itemize}
        \item login
        \item su
        \item sshd
        \item ftpd (sftpd)
        \item ...
      \end{itemize}
    \item Должны уметь читать /etc/passwd, .....
    \item Должны адаптироваться к изменениям (например аутентификация через Windows AD, или kerberos)
   \end{itemize}
   \pause
   \begin{center}
      And the answer is \pause PAM (pluggable authentication modules)  
   \end{center}
\end{frame}

\begin{frame}{Немного о PAM}
  \begin{itemize}
    \item PAM это динамическая библиотека (попробовать {\tt ldd /bin/login})
    \item Конфигурация PAM
      \begin{itemize}
        \item {\tt /etc/pam.conf}
        \item {\tt /etc/pam.d/...}
          \begin{itemize}
            \item Сервисы
            \item system\_auth
          \end{itemize}
        \item {\tt /etc/nsswitch.conf}
       \end{itemize}
    \end{itemize}
\end{frame}

\end{document}
