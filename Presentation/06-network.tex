\documentclass[ignorenonframetext, professionalfonts, hyperref={pdftex, unicode}]{beamer}

\usetheme{Copenhagen}
\usecolortheme{wolverine}

\usepackage[orientation=landscape, size=custom, width=16, height=9.75, scale=0.5]{beamerposter}	

%Packages to be included

\usepackage{textcomp}

\usepackage[russian]{babel}
\usepackage[utf8]{inputenc}
\usepackage[T1]{fontenc}

\usepackage{beamerthemesplit}

\usepackage{ulem}

\usepackage{verbatim}

\usepackage{ucs}
\usepackage{listings}
\lstloadlanguages{C, make, bash}

\lstset{escapechar=`,
	extendedchars=false,
	language=C, 
	tabsize=2, 
	columns=fullflexible, 
%	basicstyle=\scriptsize,
	keywordstyle=\color{blue}, 
	commentstyle=\itshape\color{brown},
%	identifierstyle=\ttfamily, 
	stringstyle=\mdseries\color{green}, 
	showstringspaces=false, 
	numbers=left, 
	numberstyle=\tiny, 
	breaklines=true, 
	inputencoding=utf8x,
	keepspaces=true,
	morekeywords={u\_short, u\_char, u\_long, in\_addr}
	}

\definecolor{darkgreen}{cmyk}{0.7, 0, 1, 0.5}

\lstdefinelanguage{diff}
{
    morekeywords={+, -},
    sensitive=false,
    morecomment=[l]{//},
    morecomment=[s]{/*}{*/},
    morecomment=[l][\color{darkgreen}]{+},
    morecomment=[l][\color{red}]{-},
    morestring=[b]",
}



%%%%%%%%%%%%%%%%%%%%%%%%%%%%%%%%%%%%%%%%%%%%%%%%%
%%%%%%%%%% PDF meta data inserted here %%%%%%%%%%
%%%%%%%%%%%%%%%%%%%%%%%%%%%%%%%%%%%%%%%%%%%%%%%%%
\hypersetup{
	pdftitle={Введение в GNU/Linux},
	pdfauthor={Epam/LLPD}
}





%%%%%% Beamer Theme %%%%%%%%%%%%%

	
\title{Введение в GNU/Linux}
\author{Epam/LLPD}



%%%%%%%%%%%%%%%%%%%%%%%%%%%%%%%%%%%%%%%%%%%%%%%%%
%%%%%%%%%% Begin Document  %%%%%%%%%%%%%%%%%%%%%%
%%%%%%%%%%%%%%%%%%%%%%%%%%%%%%%%%%%%%%%%%%%%%%%%%




\begin{document}

\frame{
	\frametitle{Сетевые интерфейсы Linux}
	\titlepage
	\vspace{-0.5cm}
	\begin{center}
	%\frontpagelogo
	\end{center}
}
\frame{
	\tableofcontents
%	[hideallsubsections]
}

\begin{frame}{Что нужно реализовать}
  \begin{itemize}
    \item Возможность устанавливать ip адреса различным сетевым интерфейсам
    \item Возможность настраивать таблицу маршрутизации
    \item Возможность настраивать разрешение имен (DNS)
    \item Возможность соединять разные сетевые интерфейсы разными способами
  \end{itemize}
\begin{itemize}
\item
\end{itemize}
\end{frame}

\begin{frame}{Нижний уровень: команды управления настройками сети}
\begin{itemize}
  \item ifconfig
  \item iproute2
\end{itemize}
\end{frame}

\begin{frame}{Словарик базовых команд}
  \begin{small}
  \begin{tabular}{|l|l|l|}
    Действие & ifconfig/route & iproute2 \\
    Включение интерфейса & {\tt ifconfig <iface> up} & {\tt ip link set <iface> up} \\
  Выключение интерфейса & {\tt ifconfig <iface> down}  & {\tt ip link set <iface> down} \\
          Добавление адреса & {\tt ifconfig eth0 192.168.1.17 netmask 255.255.254.0 up} & {\tt ip addr add 192.168.1.17/23 dev eth0} \\
     Добавление дополнительного адреса & {\tt ifconfig eth0:1 10.9.8.2 netmask 255.255.255.0 up} & {\tt ip addr add 10.9.8.2/24 label eth0:1 dev eth0}\\
     Добавление дефолтного пути & {\tt route add default gw 192.168.1.1} & {\tt ip route add default via 192.168.1.1} \\
  \end{tabular}
\end{small}
\end{frame}
\begin{frame}{Упражнение}
\end{frame}
\begin{frame}{Конфигурация сети в Altlinux, RedHat, Fedora}
  \begin{itemize}
    \item {\tt /etc/sysconfig/network}
    \item {\tt /etc/sysconfig/network-scripts}
    \item {\tt service network restart}
  \end{itemize}
\end{frame}
\end{document}
