% common part for every lection

\documentclass{beamer}

\usetheme{Warsaw}
\usefonttheme[onlylarge]{structurebold}
\setbeamerfont*{frametitle}{size=\normalsize,series=\bfseries}
\setbeamertemplate{navigation symbols}{}

\usepackage{pdfpages}
\usepackage{soul}
\usepackage{ucs}
\usepackage[utf8x]{inputenc}
\usepackage[TS1,T2A]{fontenc}
\usepackage[english,russian]{babel}
\usepackage{times}
\usepackage{listings}

\author[Author, Vlad Shakhov]{Влад 'mend0za' Шахов\\Linux \& Embedded Team Leader}

\institute[SaM Solutions]
{
  Linux \& Embedded Department
}

\date[Dec 2012]

\subject{Linux QA training}

\pgfdeclareimage[height=1.5cm]{sam-solutions-logo}{clipart/sam-solutions-elinux}

\logo{\pgfuseimage{sam-solutions-logo}}

\graphicspath{{./clipart/}}



\title[SaM Solutions. Linux QA Training]
{
  Часть 4.\\
  Обработка текста.
}

\begin{document}
\begin{frame}
  \titlepage
\end{frame}

\begin{frame}{Введение в Unix-Way}
  
  Заповедь номер 3:\newline 
  \begin{center}
    \alert{\Huge{Всё есть текст}}\footnote{cм \href{http://www.freesource.info/wiki/Stat'ja\_Klassicheskijj\_Unix\_Way}{статью Дениса Смирнова 'Классический Unix-way'} }\newline
  \end{center} \pause
  
  Заповедь номер 4:\newline
  \begin{center}
    \alert{\Huge{Пускайте данные по трубам}}\newline
  \end{center} \pause

  Заповедь номер 5:\newline
  \begin{center}
    \alert{\Huge{Всё есть файл}}\newline
  \end{center}

\end{frame}

\begin{frame}{Текст в Unix-Way}
  В Unix (и Linux) в виде обычного текста или \alert{plain text} представлены:\pause
  \begin{itemize}
    \item \alert{конфигурационные файлы}, как локальные\footnote{в каталоге \$HOME} , так и общесистемные\footnote{в каталоге /etc} \pause
    \item \alert{системные логи}\footnote{справедливо для \alert{syslog} и совместимых систем}
    \item \alert{исходные тексты программ}, включая скрипты на Shell
    \item \alert{основной формат ввода и (или) вывода данных} для множества программ и утилит
  \end{itemize} \pause

  Богатый выбор изощрённых острозаточенных инструментов для работы с текстом во всех представлениях\footnote{'Unix is toolbox' - 'Unix это ящик с инструментами'}

\end{frame}

\section{Операции над текстом}


\begin{frame}{Просмотр обычных файлов}
  %\frametitle{Просмотр обычных файлов}
  \begin{itemize}
    \item Просмотр текста:
      \begin{itemize} 
	\item \alert{cat} - вывести на stdout\footnote{Для двоичных файлов: чревато порчей настроек терминала} \pause
	\item \alert{more} - вывести, разбив на страницы
	\item \alert{less}\footnote{может отсутствововать в стандартной поставке} - \alert{more} на стероидах, с прокруткой, поиском 
      \end{itemize} \pause
    \item Просмотр двоичных данных:
      \begin{itemize} 
	\item \alert{od} - дамп файла в не-текстовых форматах
\lstinputlisting[frame=single,basicstyle=\tiny]{samples/od}
        \item \alert{strings} - извлечь текстовые строки из двоичных файлов
      \end{itemize}
  \end{itemize}

\end{frame}


\subsection{Редактирование}
\begin{frame}{Редакторы}

  \begin{itemize}
    \item  Любой редактор, с которым вы можете справится. \pause Но его может не быть в вашей системе\ldots \pause
    \item Редактор \alert{vi} присутствует как стандартный в любой Unix-подобной системе\footnote{В этом качестве он внесен в стандарт Single Unix Specification. Существует множество реализаций редакторов, совместимых c vi: vim, elvis, nvi, vi-mode в Emacs, Sublime Text 2, vi из busybox и т.д.} \pause
    \item Не обязательно редактировать локально: В \alert{mc}, \alert{vim}, \alert{emacs} есть возможность удалённого редактирования файлов\footnote{С получением и сохранением данных по протоколу FTP и SSH}.
  \end{itemize}

\end{frame}

\begin{frame}{vi и vim}
  Перед стартом:

  \begin{enumerate}
    \item  Редактор vi изначально создавался как универсальный и переносимый\footnote{Обязан работать на любых типах терминалов и виртуальных консолей}. Все действия можно осуществить на алфавитно-цифровой части клавиатуры, без мыши. \pause
    \item \alert{Редактор командного стиля}\footnote{Командного стиля, а не меню-ориентированный}. Действия подачей прямых управляющих команд. \pause \newline
      3 основных режима: \sout{портить текст и противно бибикать}
      \begin{itemize}
	\item[-] \alert{Командный режим} (Normal mode) - по умолчанию при запуске.
	\item[-] \alert{Режим изменения текста} (Edit mode)
	\item[-] \alert{Режим построчного редактирования} (Ex mode) - операции над файлом целиком\footnote{сохранение, открытие файлов, выход, вставка файла в текущий и т.д.}.
      \end{itemize}
  \end{enumerate} \pause
  \alert{Упражнение}: проходим \alert{vimtutor}, встроенный в vim учебник\footnote{ export LANG='ru\_RU.UTF-8' - на русском языке }
\end{frame}

\subsection{Программы-фильтры}

\begin{frame}{Текстовый фильтр}

  Определение:\newline \alert{Текстовый фильтр} - программа, обрабатывающая и преобразующая текст. \newline

  Примеры: \alert{sort}, \alert{cat}, \alert{tac}, \alert{rev} \pause
  \begin{itemize}
    \item Фильтр, запущенный без параметров - читает стандартный ввод. 
    \item Параметры фильтра - интерпретируются как имена файлов
    \item Ключи фильтра - управляют режимами работы
  \end{itemize} \pause

  Фильтр почти всегда используется совместно с перенаправлением ввода-вывода Shell (особенно '|', pipes).

\end{frame}

\begin{frame}{Простые текстовые фильтры}

  Соглашения о параметрах: \alert{'-'} как имя файла обозначает стандартный ввод.

  \begin{itemize}
    \item \alert{cat} и \alert{tac} - вывести файл целиком \pause
    \item \alert{head} и \alert{tail} - вывести начало и конец файла \pause
    \item \alert{sort} и \alert{uniq} - сортировка и убрать повторы в отсортированном \pause
    \item \alert{grep} - поиск по образцу \pause
    \item \alert{paste} - объединить файлы построчно \pause
    \item \alert{wc} - счётчик строк, слов и байт в тексте \pause
    \item \alert{tee} - копирует стандартный ввод в файл и на экран 

  \end{itemize}
\end{frame}



\begin{frame}{Простые фильтры - упражнения}
  \alert{Упражнение 1}: посчитать сколько файлов в папке /bin\footnote{подсказка - \alert{wc} и \alert{ls}}

  \alert{Упражнение 2}: сколько слов в первых 15 строках .basrc\footnote{подсказка - \alert{head} и \alert{ls}}

  \alert{Упражнение 3}: найти, в каких файлах (и сколько их вообще) в области системных логов (каталог \alert{/var/log}) была записана информация о входе вашего пользователя в систему. \footnote{подсказка - \alert{grep} и, возможно, но необязательно, \alert{uniq}}

  \alert{Упражнение 4}: сколько было входов в систему от имени вашего пользователя? в какое время был первый? последний? \footnote{подсказка - \alert{grep}, \alert{head}, \alert{tail}}

  \alert{Упражнение 5}: то же, что и 4. Только для выходов. Дополнительно сохранить записи о ваших выходах в отдельный файл. \footnote{подсказка: + \alert{tee}}





\end{frame}

\begin{frame}{Изощрённые фильтры }
  \begin{itemize}
    \item \alert{diff} (и \alert{diff -u}\footnote{Формат \alert{diff -u} : индустриальный стандарт пересылки списка изменений между версиями текстовых данных.}) - сравнить 2 файла и получить patch (изменения между ними)
    \item \alert{patch} - утилита применения изменений от diff \pause
    \item \alert{sed} - не-интерактивный поточный редактор текста 
    \item \alert{awk} - язык и утилита сканирования и обработки текста
  \end{itemize}
\end{frame}


\section{Поиск в тексте. Регулярные выражения}

% file /bin/*|grep symbolic

\end{document}
