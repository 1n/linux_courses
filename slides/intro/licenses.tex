\begin{frame}{Лицензии: открытые и свободные}

	\begin{block}{ Р.Столлман: 4 свободы}

		\begin{itemize}
			\item Свобода 0: Свобода запускать программу в любых целях.
			\item Свобода 1: Свобода изучения работы программы и адаптация её к вашим нуждам. 
				Доступ к исходным текстам является необходимым условием.
			\item Свобода 2: Свобода распространять копии,  так что вы можете помочь вашему товарищу.
			\item Свобода 3: Свобода улучшать программу и публиковать ваши улучшения,
				так что всё общество выиграет от этого.
				Доступ к исходным текстам является необходимым условием.
		\end{itemize}
	\end{block}


\end{frame}

\begin{frame}{Copyleft }

	\begin{block}{ \textcopyleft  -- ``Копилефт''}
	Авторское лево -- концепция и практика использования законов авторского права для обеспечения 
	невозможности ограничить любому человеку право использовать,  изменять и распространять как 
	исходное произведение,  так и произведения,  производные от него.
	\end{block}


	При копилефте все производные произведения должны распространяться под той же лицензией,
	что и оригинальное произведение.

\end{frame}


\begin{frame}{Лицензии}
	\begin{itemize}
		\item GPL
		\item LGPL
		\item AGPL
		\item BSD
		\item MIT
		\item Mozilla Public License
		\item Apache Software License
		\item Creative Commons *
	\end{itemize}
\end{frame}
